Password file disclosures are a frequent problem for many companies, which
makes their users the target of identify theft and similar attacks.   This 
work provides a new general cryptographic technique called PolyHashing that 
protects hash information unless a threshold of correct hashes are known.
In this work, we apply PolyHashing to the problem of password verification and 
storage to build a system called PolyPassHash. % PolyPassHash prevents 
%efficient password cracking in the event of a password file disclosure.  
Even if the password file and all other data on disk is obtained by a 
malicious party, the attacker cannot crack any individual password without 
simultaneously guessing a large number of them correctly.   PolyPassHash
is the first single server, software-only technique that increases
the attacker's search space exponentially.   The result is that even cracking 
small numbers of weak passwords is infeasible for an attacker.   

%For sufficiently 
%strong passwords, PolyPassHash will even provide information theoretic 
%security, making password cracking impossible.  
PolyPassHash achieves these properties with similar efficiency, storage,
and memory requirements to existing salted hash schemes,
%performance.
%PolyPassHash requires comparable storage to the current best practice salted 
%secure hashing scheme, requiring a total of about 1KB of additional memory 
%(with no additional per account cost) and no more than 1 extra byte of disk 
%space per account.
%Our implementation of PolyPassHash is similar
%in speed to existing salted hash schemes, 
performing
tens of thousands of account authentications per second.    
When using the current best practice (of salting and hashing), 
cracking three passwords that are comprised of 6 random characters on
a modern laptop would take under a hour.  However, when protected with
PolyPassHash, cracking these passwords would take longer than the estimated 
age of the universe even when using every computer on the planet.
%\cappos{Not the strongest thing to end on.   Consider talking about exponential
%increase in time last.}
