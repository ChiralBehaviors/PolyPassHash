\section{Related Work} \label{sec-related}

%\cappos{ cite: \url{http://cs.gmu.edu/~csnow/library/unix/Klein_passwd.pdf} 
%and lots of other things}

%\cappos{A nice survey paper with related work details~\cite{tsai2006password}.}
There has been extensive work on password security stretching back many
years~\cite{morris1979password,klein1990foiling, florencio2007large}.   
Password database disclosure is a problem that, if anything, seems to be
more prevalent as time goes on~\cite{clair2006password,miranteTR13,passwordresearchblog}.
Password security has been the focus of much study with many promising 
solutions solving different portions of the problem~\cite{tsai2006password}.

Our work on PolyPassHash is unique in that is the first password database
protection scheme that:
\begin{enumerate}
\item assumes the attacker can read all persistent storage,
\item requires only a software change on the server,
\item and requires exponentially more time for the attacker to crack passwords.
\end{enumerate}

PolyPassHash can be deployed with minimal server changes and without modifying 
clients at all.   

{\bf Multi-server Password Authentication.}
There are a wide variety of authentication schemes that use multiple servers
to store password data~\cite{Chai20071046,bagherzandi2011password, katz2005two}.
The assumption is that the attacker cannot compromise a threshold of
the servers.  In contrast, PolyPassHash uses a single server but uses
a threshold system to hide information that can only be unlocked
with a threshold of correct user passwords.

{\bf Decoys.}
Recently, researchers have suggested multiple techniques that use a set of 
extra password entries~\cite{Kontaxis_CCS_2013, juels2013honeywords}.  For 
example, the Honeywords~\cite{juels2013honeywords}
system uses a separate server holds information about which password entry is 
correct.   If an 
attacker obtains the password database then they do not know which password 
entry is correct.   Entering a password which matches the hash of a 
different password entry will trigger an alarm which notifies the 
administrator of a password hash file breach.   However, for this 
to work, there must be a separate, secure server which authenticates 
the index of that entry in the file (a one byte value).   
The real value of HoneyWords is that it can also operate when that server is
offline and / or check passwords at a later time to detect breaches later.
%If it is 
%possible to build such a system, it is not clear why that server 
%cannot simply securely store and check the password hash (a fixed 
%size value of about 32 bytes).  
PolyPassHash utilizes ideas from these works in constructing the partial
verification technique discussed in Section~\ref{sec-partial}.

{\bf Key Stretching.}
One way to mitigate the effectiveness of password hash cracking is to
use techniques for key stretching~\cite{kelsey1998secure}.   This involves
performing multiple rounds of cryptographic operations to validate a key.
This effectively slows down both the attacker's cracking of passwords and the
user's authentication by the same factor.   In contrast, PolyPassHash results 
in an exponential increase in the amount of time needed by requiring multiple
passwords to be simultaneously guessed.  Key stretching is orthogonal
to PolyPassHash and could be trivially used in conjunction.
%that knows a threshold of correct account authentications.


%{\bf Hardware-based Password Database Encryption.}
%Several companies use password database protections that center
%around storage of keys or passwords in hardware, such as through a USB dongle.
%Hardware based solutions are fundamentally more 
%secure than PolyPassHash since even if an attacker has access to all memory,
%the password database is not at risk.   
%Unfortunately, these systems are not widely deployed.
%One issue they face is that each device presents its own interface for password
%storage and decryption.   A company that wants to deploy a solution must 
%provide new authentication code for the specific device and also purchase
%the devices.   
%If the hardware token stops workting, the data may be irrevocably lost.
%If the server storing a device is damaged or crashes, it 
%is not possible to bring up a backup using the same authentication 
%database.  On the other hand, PolyPassHash is a purely software-based solution.
%This not only makes it easier to deploy, but also means that a backup can be 
%brought online in the same way that the normal server boots.   As a result
%PolyPassHash is much easier to deploy and manage in distributed computing
%environments such as in cloud computing.

%\cappos{Read over and add cites.}

{\bf Bounded Attacker.}
Di Crescenzo~\cite{di2006perfectly} proposed a scheme for protecting
password data when an attacker can only read a bounded amount of data from 
storage.   This works by an organization configuriing
network monitoring hardware and setting up a separate server to process 
authentication requests.
In the widely published password file compromises, the 
attackers were able to steal complete password file 
data~\cite{miranteTR13,passwordresearchblog}.   In contrast, 
PolyPassHash requires no network changes or monitoring and works even when an 
attacker has complete access to stored information, such as a disk backup.

Prior work by Gwoboa~\cite{gwoboa1995password} hides passwords using
a trapdoor function (public key cryptography) and techniques from threshold
cryptography.   It can authenticate users with two hidden pieces of 
information, a user ID (likely not the user name for security reasons) and 
the password.   However, a major concern of the scheme is how the private
key is stored on the server.   The authors propose splitting it amongst 
multiple systems and using threshold cryptography.   Clearly if all 
persisted data is known, this key (and thus all passwords) are at risk.
In PolyPassHash, as long as a threshold of passwords are not known, all 
persisted data can be stolen by an attacker without compromising user 
passwords.


{\bf Biometrics.}
Biometric authentication has substantial promise for secure
authentication~\cite{atallah2005secure, snelick2005large, tuyls2004capacity, 
boyen2005secure, erkin2009privacy, kerschbaum2004private, osadchy2010scifi,
monrose2001cryptographic, sae2012biometric}.
There has been a substantial amount of work on how to store and authenticate 
users with this information.   Like PolyPassHash, some of these systems 
use a threshold of information to validate and authenticate users, in part
to deal with noisy biometric data~\cite{juels2006fuzzy, ballard2008practical}.  
While users must still remember a password to use PolyPassHash, it does not 
require client-side hardware.

Prior work uses keystroke dynamics to change stored password 
data~\cite{monrose2000keystroke}.  This technique relies on reading timing 
information from when the user types their password into a site.   This 
provides promising protections, but requires changes to the client and server 
to correctly operate.   In comparison, PolyPassHash protects password
hash information with no change to the client and minimal server changes.

{\bf Authentication Using Tokens or Smart Cards.}
Much authentication has looked at authentication in the context of
banking~\cite{deo1998authentication, yeh2010two}, health 
services~\cite{ahn2002towards},
or a more general context~\cite{chien2002efficient, yang1999password}.   These 
systems are
extremely effective and are widely used for banking and protecting access
to classified systems.   Unfortunately, these devices incur a per-user 
cost and thus are not often used in contexts where the user and server have
no prior commercial or authentication relationship.   PolyPassHash can 
be applied to webmail systems and social networks where this relationship
does not exist a priori.

{\bf Two-Factor Authentication.}
The use of two-factor authentication~\cite{di2005two} is provided by some 
popular services (typically through a user's smartphone).   The use of 
two-factor authentication does not
change PolyPassHash's use in any way.   Users can easily get the best of both
protections by a simultaneous deployment of each technology.



{\bf Multiparty Computational Authentication.}
There are a variety of schemes that perform secure, remote authentication using
computation by the client and server on legacy hardware~\cite{wu1998secure,
Lomas_SOSP_89, chien2001modified, jan1998paramita, gong1995optimal, 
camenisch2010credential, brainard2003nightingale,
katz2001efficient, katz2003forward, gong1993protecting}.  These schemes have 
significant positive aspects such as (in some cases) requiring an attacker to 
be online to validate communications.   However, they require multiparty 
protocols which require changes on clients and servers.   They also do not 
function in non-distributed scenarios.   PolyPassHash works with no 
changes to clients, minimal visible changes to the administrator and
operates on a single system.


{\bf Related Key Exchange Schemes.}
There are also many systems for secure key exchange~\cite{shoup1999formal} 
such as Pass\-word-Au\-then\-ti\-cated Key Exchange
(PAKE)~\cite{boyko2000provably, shen2010towards,sathik2010secret, 
jablon1996strong}, Encrypted Key Exchange (EKE)~\cite{steiner1995refinement,
lucks1998open, jablon1997extended}, and further 
enhancements~\cite{wang2005strengthening}.   These 
systems allow parties that
share a password to securely find an encryption key to hide communications.   
These systems provide excellent protections and can handle compromises
of the memory of systems in some cases.   However, unlike PolyPassHash, they 
typically involve multi-round authentication and require changes to both the 
client and server.


{\bf Helping Users Choose Stronger Passwords.}
There have been many efforts to help users to choose stronger,
more memorable passwords or expose weak 
passwords~\cite{topkara2007passwords,klein1990foiling,bishop1995improving,
schechter2010popularity,
komanduri2011passwords, shay2010encountering, xkcdpassword}.   These 
techniques can be very effective at protecting users,
but must be adopted by users.
PolyPassHash provides an exponential improvement in protection for user 
passwords.   Assuming threshold passwords are strong
(which these methods assist with), PolyPassHash strengthens the
protection of stored user passwords.   This is especially critical when
partial verification is used.

{\bf Password Managers.}
There are a myriad of password managers that help users choose secure
per-site passwords including LastPass, 1Password, and OnePass.   These
systems store password data and lock it using a user's credentials.   As a
result, it is the case that the third party software (and often their server)
will know the user's password.

To mitigate this, several groups have proposed cryptographic techniques
to take a user's password and generate secure, per site 
passwords~\cite{halderman2009lest,ross2005stronger, halderman2005convenient}.
These techniques are effective (and more secure) but can create passwords that 
are incompatible with the server's password policy.

These techniques require client side changes (only) while PolyPassHash
requires only server side changes.   Both can (and should) be used safely 
in conjunction for improved security.


{\bf Single Sign-On.}
Single sign-on systems like OpenID and OAuth have promise for organizations to
securely offload authentication
to a third party.   This proves convenient for users, but is far from 
ubiquitous for a variety of reasons~\cite{sun2010billion}.  These
systems have some security issues~\cite{openidsecurity,oauthsecurity}, but
overall can be effective when properly used.   PolyPassHash integrates 
cleanly with non-administrator logins for such systems.   In addition,
PolyPassHash can be used by the Single Sign-On provider to provide 
security to the authenticating users.

{\bf Non-password Authentication.}
Many researchers have proposed authentication based upon non-password items
such as pictures~\cite{dhamija2000deja}.   In practice, these
systems can have security limitations if users do not appropriately choose
their authentication tokens~\cite{davis2004user}.   For exotic 
authentication mechanisms like this, PolyPassHash functions well for 
non-administrator accounts, requiring no changes to the system.



% JAC: It looks like ``A New Threshold Password Authentication Scheme'' isn't
% available.   


%\cappos{There are schemes where people do crypto operations~\cite{hopper2001secure}.   Totally crazy!}

