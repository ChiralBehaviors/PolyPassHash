\section{Background}
\label{sec-background}

This work requires two fundamental building blocks, a 
salted hash function and a $(k,n)$-threshold scheme.   While many
different schemes could be utilized, for concreteness we will describe our
system using SHA256 for the hash and Shamir Secret Sharing for the 
threshold scheme.

\subsection{Salted SHA256}

A cryptographic hash function, such as SHA256, is one where (with extremely 
high probability) 
each hash input produces a fixed size output where it is impractical to 1) 
find another value that produces the same output and 2) discover the input 
given an output.  This is useful for situations like password checking because 
the hash of the password can be stored instead of the password.   If 
the password file is compromised, then the attacker must be able to find the 
input for that output.   

However, in practice many users choose the same set of passwords, allowing the
easy construction of lookup tables with the hashes of common 
values~\cite{rainbowtable}.  To mitigate this problem, current best practice for
password protection schemes is to also use a randomly generated value called a 
salt.   The salt is created at the time the account is generated and this value
is included in the password when it is hashed.   In this way the same password 
will have different hash values for different users, because the salt will be 
different.   The salt is typically stored in plaintext along with the 
hash so that the authentication function can use the same salt value when 
authenticating the password.


\subsection{Shamir Secret Sharing}

Shamir Secret Sharing~\cite{shamir1979share} is an algorithm where a
secret is divided into a set of $n$ shares.   These shares are typically
provided to different parties.
If a threshold $k$ (specified when the secret is divided) of those are 
collected, the secret can be reconstructed.   To hide a secret, Shamir Secret 
Sharing computes $k-1$ random coefficients for a $k-1$ degree polynomial 
$f(x)$ in
a finite field (commonly GF-256 or GF-65536).   The $k$th term (commonly the
constant term) contains the secret.  
 To compute a share, a value between 1
and the order of the field is chosen.   The polynomial is evaluated 
with $x$ equal to the share value.   The terms $x$ and $f(x)$ are used as the
share.   
To reconstruct the secret from at least $k$ shares, a party can
interpolate the values in the finite field to find the constant term (i.e. the 
secret).   In practice, interpolation is often computationally optimized so 
that {\bf only} the constant term is recovered.

For example, suppose one wishes to hide a secret $235$ so that it can only be
reconstructed if $3$ shares are provided.   The person hiding the secret can 
choose random terms and build a GF-256 polynomial such as $f(x) = 24x^2 + 
182x + 235$.   Shares can then be generated by computing $x$ and $f(x)$ such
as: $(1,185)$, $(2, 183)$, $(3, 229)$, $(4,67)$, etc.   A party that has at 
least three shares can interpolate to reconstruct the full polynomial of
$f(x)$ and thus the secret ($235$).

If one requires the ability to generate additional shares
after the original share creation, as in this work, a different recontruction
procedure is applied.   To do this, Lagrange interpolation of the secret during 
reconstruction is performed instead of simply computing the last term.   This 
makes share recovery slightly more computationally
complex, but it is then possible (and efficient) to generate additional shares
simply by evaluating $f(x)$ for the specified share.

In many cases, the desired secret will be larger than the size of the finite
field.   To store this secret, one can break it into pieces that are the size
of the finite field (often one byte) and apply the above technique separate on 
each piece.  In this case, the same share number $x$, is typically used for 
each share $f_i(x)$.

When given a set of any $k$ distinct shares, whether valid or invalid, Shamir
Secret Sharing will produce a polynomial of the appropriate length.
This means if any share is invalid, the resulting polynomial will be 
incorrect.   To avoid this issue, 
typical implementations of Shamir Secret Sharing also store an integrity check
to detect if an incorrect share has been provided.   This may be
done by including some portion of the secret (typically a few bytes at the 
end) that represent a hash of the decoded value.  While the techniques
in this paper function identically regardless of whether or not this
hash exists, we assume its existence to prevent erroneous data from
being used as valid.   
%additional information (assurance that a validated 
%can utilize shares hashed in this manner
%to an attacker.   By providing a hash, an attacker will know with a high 
%degree of certainty when they have provided $k$ correct shares.   As a result, 
%PolyHashing will choose to eschew this integrity verification for stronger 
%security.




