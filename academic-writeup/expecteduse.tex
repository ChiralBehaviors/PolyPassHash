\section{Discussion}
\label{sec-expecteduse}

This section discusses strengths, limitations, and expected uses of 
PolyPassHash as compared to other techniques.


{\bf How does PolyPassHash compare to having an administrator enter a key
at boot time to unencrypt the password database in memory?}

With the administrator entering a key, the system cannot validate passwords 
upon startup until an administrator intervenes.  With partial verification, 
PolyPassHash can process user authentications immediately.   Furthermore, 
if the administrators enter a key, all system administrators would likely 
share this key to be able to quickly act to restart the system.   From 
a security design standpoint, it is considered bad practice to share
passwords.  With 
PolyPassHash different administrators do not need to share a password or key.
Each administrator may have their own password allowing access control to 
happen naturally as administrators join and leave the organization.


{\bf Why not store a key to decrypt the password database in hardware instead?}

Storing a key in trusted hardware (such as a USB 
dongle~\cite{passwordhardwaredongle}) has security benefits,
but also has significant deployment challenges.  
%The software must be rewritten to interface with that 
%manufacturer's secure token.   
If the hardware stops working, the protected data is 
likely to be irrevocably lost.  %Furthermore, 
To have backup servers, it is essential to have identical 
duplicate copies of the secure token.   This complicates use in scenarios 
like cloud computing or even just a standard master / slave deployment.
On the other hand, PolyPassHash is entirely software based and a backup system
can be brought online in the same way that a normal server boots.

{\bf How does an attacker that can read arbitrary memory on the 
authentication server impact PolyPassHash?}

If an attacker can read arbitrary memory on a running server where the 
threshold of authentications has been met, the attacker can recover
the Shamir Secret Store.   This allows the attacker to recover the 
hashes for each password.   An attacker could then crack passwords
individually in the standard way.    While these are not the majority of
disclosed database compromises~\cite{passwordresearchblog,miranteTR13}, 
neither PolyPassHash nor techniques like hardware key storage 
are significantly helpful in this situation.
%protect against situations where the attacker has administrator access on a 
%running authentication system.

However, if an attacker can read
arbitrary memory, the attacker can gain access to plaintext passwords
as they are provided by clients.    This is because typically the client
provides the server with the password, which the
server then combines with the salt to get the hash.  
So without changes to the client software, any server-side
password storage technique will leak plain-text passwords to an attacker
that can read arbitrary memory on a running server.



{\bf How does PolyPassHash change how users interact with existing password authentication systems?}

Using PolyPassHash results in very minor changes to existing password
authentication systems.   
%The only user perceptible difference is when 
%the system is restarted, authentication attempts will be delayed or 
%declined until the threshold is reached.   
%
The client tools for password authentication do not change in any way.   In 
fact, because PolyPassHash only impacts password storage, it is invisible to 
clients.   %A user cannot tell if PolyPassHash is being used, much like it is 
%not possible for a user to tell if a server stores a password using a salted 
%secure hash or in plaintext.
For the administrator's toolchain, the only change is that 
when an account is created, the administrator can specify if the account counts
toward the threshold or not.
%Note that in many cases, such as cloud services, the 
%separation for threshold and thresholdless accounts is clear.   For example, a 
%web service may choose to have a command line interface used by administrators
%to create accounts with a threshold, while the web interface will only
%create thresholdless accounts.

{\bf How does using PolyPassHash integrate with existing password storage 
formats?}

The file format for password storage is similar to existing systems.
For systems which use a database, one can simply insert the share number
as a new column in the table and change the authentication logic to include
PolyPassHash.   

However, 
many servers (such as Linux, Mac and BSD) use the {\tt /etc/shadow} or 
{\tt /etc/master.password} file 
colon delineated formats.
%by `:') such as the username, password expiry date, last date the password was 
%changed, and a single `password' field.
The `password' field contains both the salt and hash but these
different portions of the field are not 
delineated.  
%Instead the storage and verification code knows which 
%bytes of the `password' field are used for each task.   
For PolyPassHash, one can also add the share number at a known byte location
within the `password' field (such as the beginning or end), making this
variable length, opaque string one byte longer.
%We do not anticipate issues with this since the `password' field and length
%are opaque.  (In fact administrators commonly change the length and content of
%the `password' field manually.)   Other fields are separated from the password
%field using `:' as a delimeter and so will not be impacted.


{\bf If there is a single administrator, is PolyPassHash useful?}

There is value in PolyPassHash whenever there are multiple
accounts, even if only one is an administrator.   If an attacker steals the
password file, but does not know the password for the administrator account, 
the attacker cannot crack other user accounts.   
Assuming that the attacker does not know the administrator password
from other sources, such as its use on multiple sites, the 
attacker cannot crack a thresholdless password without cracking 
the administrator password.


{\bf What threshold values are likely to be used?}

%It is logical to assume that most policies for login will request that
Ignoring partial verification,
$k$ of $n$ administrators (or power users) must login to unlock the
PolyPassHash store.   The value of $n$ is largely irrelevant
except it must be smaller than the field size since there are only $n-1$ 
Shamir Secret shares.   (Note $n$ only includes accounts that should count
against the threshold.)
%
%The primary value that matters is $k$, the threshold setting.   
%The threshold value $k$ is the number of individuals 
%that must enter their password before unlocking a data store.   
%A very large value may impact the availability of the system since logins 
%cannot be processed until the threshold is reached.   
We believe that most organizations will have a small 
value for $k$, such as $1$ to $5$.   As our results later show, 
when coupled with non-trivial passwords, even a small value of $k$ 
increases password strength immensely.



{\bf When is PolyPassHash most (and least) valuable?}

When only a single user account exists, such as on a smartphone, 
there is no benefit to using PolyPassHash on the local device.   However, 
essentially all devices are networked and interact with servers.   Web and 
cloud services benefit from PolyPassHash as do any devices that support
multiple user logins.   
Services that manage logins from multiple parties, substantially benefit from
PolyPassHash.     This is because PolyPassHash makes 
a password file compromise only impactful if a threshold of administrator
passwords are already known.   %We believe that any organization that has 
%passwords from multiple users and wants to protect those
%passwords (which should be a universal goal) should deploy PolyPassHash.


{\bf What happens if so many threshold users forget their passwords that a
threshold cannot be reached?}

If there are not enough known threshold users, all password data in the
password file is useless.   However, this does not mean that the device is
unusable.   If used, partial verification will still allow users to log in.   
However, it will not be possible to create new accounts since Shamir
Secret Shares cannot be generated and the AES key protecting the thresholdless
accounts is not known.  Furthermore, mechanisms like root password recovery 
through
the console will still work, allowing any data on the system to be accessed.




%{\bf Why might PolyPassHash be adopted when existing schemes are infrequently 
%used to protect passwords?}
%
%A huge advantage of PolyPassHash is that it is a software-only solution that
%requires no additional hardware or setup.  The only change is in the 
%authentication software.    Furthermore the change is not burdensome (or even
%visibile) to end users.   We believe the power of simply allowing an 
%administrator to improve security through a simple software action (like
%{\tt apt-get install polypasshash}) could substantially increase adoption.



{\bf What happens if users choose extremely weak passwords like 123456, 
letmein, and password?}

Ignoring partial verification for the moment, 
if extremely weak passwords are used for a threshold of threshold accounts 
(like administrators), PolyPassHash will not provide strong protection.   If
there are only a few bits of entropy in the password, while exponentially
larger, the search space will still be small.   It is thus critical that
the administrators choose strong passwords.   However,
if a thresholdless account uses a weak password, the attacker must first
crack a threshold of accounts, thus providing strong protection.

If partial verification is used, then some bytes of the password's 
hash are leaked.   If an extremely weak password is used, this
will be effectively leaked to the attacker.   One should think of partial
verification as reducing the password strength via the number of leaked
bits.   Thus a password with fewer bits of entropy than the partial 
verification length, can be cracked as though it is protected using today's
best practices.


Independent of all of this, extremely weak passwords are susceptable to 
guessing by an external party (without database access) and as best practices
sites should block them from use~\cite{bancommonpasswords}.




