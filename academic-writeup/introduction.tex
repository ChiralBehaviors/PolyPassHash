\section{Introduction}



Password file disclosure is a major security problem that has impacted dozens
of organizations including Hotmail, LastFM, Formspring, ScribD, 
the New York Times, NVidia, Evernote, Billabong, Gawker, LinkedIn, Linode,
ABC, Yahoo!, eHarmony, LivingSocial, and Twitter~\cite{miranteTR13}.
Security best practices advocate that 
the passwords should not be stored in plain text.   Instead a user's password 
should be subjected to a salted hash and then stored.
The (secure) hash acts as a one-way function to ensure that an attacker cannot
trivially read the passwords.   The salt is a random
value that complicates the use of lookup tables to immediately crack weak 
passwords.   
Once password data is compromised, attackers have proven adept at quickly 
cracking large numbers of passwords.    For example, in 2012 an attacker 
compromised
6.5 million LinkedIn accounts and within a week at least 60\% of those
passwords were cracked~\cite{Linkedinpassword}.  


In situations where an attacker has root access to a running system, the
attacker can intercept and read passwords from memory before any 
protections like salting and hashing apply, nullifying storage 
protections.   However, while
attackers have proven adept at stealing password databases, the most 
disclosed attack vectors, such as SQL injection~\cite{passwordresearchblog, 
miranteTR13}, do not require root access to a running authentication system.
Thus our focus in this work is on protecting against an attacker that can
read all data from persistent storage.


%Fundamentally, it is not 
%possible for current techniques to protect a user that has chosen a password 
%that is easy for an attacker to guess.   As most users choose 
%predictable passwords (the 100 most popular passwords are used 40\% of 
%the time by some estimates~\cite{10Kpassword}), it is possible for
%an attacker to quickly breach many accounts.

\eat{
% JAC: I like this, but it is too much...
Researchers have proposed a variety of different techniques to address
password hash disclosures.   This includes diverse ideas from the use of 
decoy password hashes~\cite{juels2013honeywords, Kontaxis_CCS_2013}, key 
exchange systems~\cite{boyko2000provably,steiner1995refinement}, multiparty
computation~\cite{wu1998secure,gong1993protecting}, biometric 
authentication~\cite{atallah2005secure, snelick2005large}, 
%key stretching~\cite{kelsey1998secure}, 
and multi-server
password storage~\cite{Chai20071046,bagherzandi2011password, katz2005two}.
However, these existing techniques either require changes to the server 
hardware or to clients, both of which present a barrier to adoption.
}

One possible way to protect a password hash database is to encrypt it using
a key.   However, when the system restarts, the database must be decrypted
and if the key is stored on disk, the attacker can compromise it.   At a high
level, the idea behind this work is to store protection information such as
the key in a way that it is recoverable only with a threshold of passwords.
Thus the key does not reside on disk, but instead lies within the minds of the
users (unbeknownst to them).  By leveraging a threshold system, users 
organically provide the threshold of correct passwords (typically 2-4) 
needed when logging in.   


%Some security researchers have suggested more computationally expensive 
%hashing primitives as a way to rate limit an attacker's guesses (and thus 
%limiting the damage from a successful attack).   
%\cappos{Need a cite here.   I guess this counts...
%https://password-hashing.net/index.html}
%While an expensive hashing primitive would do so, it would also slow down
%login processing for sites in the normal case.   Depending on the rate of 
%decrease, this may harm their usability and be unappealing for adoption.   
%The fundamental technique of slower hashing provides only
%a linear decrease in the attacker's rate of cracking user accounts.

%One success story is the use of smart cards~\cite{chien2002efficient} and 
%similar devices for
%authentication.  These secure devices are effective in authenticating users, 
%but they require substantial changes to the way authentication systems 
%commonly work (namely, distribution of a hardware token).   This has made them 
%mostly be used for only specialized operations such as access to classified 
%records or online banking in some countries.   
%These forms of authentication to systems that have 
%many users which they know little to nothing about.   This limits their
%applicability to widely used services such as Facebook and Google.


To accomplish this, we devise a general cryptographic approach to validate 
the integrity of
information based upon a novel technique called PolyHashing\footnote{The
prefix `Poly' in this context is used as a synonym for `multi'.  So the name 
PolyHashing refers to the need for multiple hashes.}.  Previous
schemes have used threshold cryptography to hide password data across multiple 
servers.   PolyHashing uses threshold cryptography to obscure different 
stored hashes on a single server so that hashes can only be checked if a 
threshold are known.   Thus, if an attacker does not know a threshold of 
correct hashes for the system it is not possible to validate 
any hash value.   In essence, a PolyHashing store provides confidentially 
of all stored data until a certain amount of valid data is known.   Once this
threshold of data is known, a PolyHashing store then provides integrity 
checking for all future data.

Using PolyHashing, we built a system called PolyPassHash that 
protects password hashes.  PolyPassHash ensures that a party cannot 
read or check any individual password in the password list, without knowing
a configurable threshold number of passwords (often two to four).   Thus when 
a system using PolyPassHash restarts, a threshold of users must provide correct
passwords before any may be authenticated.   (We discuss an extension called
\emph{partial verification} that allows verification to occur immediately
upon reboot while handling erroenously or maliciously incorrect passwors
in Section~\ref{sec-partial}.)
%Note that some passwords,
%such as that of an administrator, can count for multiple passwords within
%this threshold.  
Note that an extension also exists to cause untrusted users (such as 
Facebook or Gmail) to have their password not count toward the threshold for
validation purposes.    

Once the threshold is 
reached, it is possible to quickly and efficiently check all future password
requests.  
In the situation where an attacker does learn a threshold of passwords, the 
remaining passwords are 
protected in as strong of a manner as existing salted hashing 
techniques.  For an attacker that does not know a threshold of passwords, 
this makes password cracking infeasible.

PolyPassHash only requires changes on one party (the server) and can be 
integrated smoothly into existing systems and verification processes.  % Apart
%from waiting for a threshold of passwords for authentication when the system
%is initialized, 
When adding PolyPassHash to an existing system, no changes are visible to 
the user.  In fact, existing
client password login mechanisms do not need to be modified in any way.   The
only changes occur in the way that the encoded password data is generated,
stored, and verified by the server.


PolyPassHash relies on simple primitives that are efficient from a storage, 
memory, and computational standpoint.   The additional storage cost is 
approximately one extra byte of data per user account over a salted 
hash.   The memory overhead of PolyPassHash is about a kilobyte of 
information, independent of the number of passwords stored.
The computational primitives are fast and well known
such as XOR, a salted hash function (SHA256), AES, and Shamir Secret 
Sharing~\cite{shamir1979share}.
As such, even when run on a three year old laptop, PolyPassHash can process 
tens of thousands of authentication requests per second.


This work presents the first password protection scheme that 1) protects 
against an attacker that can read all persistent storage on the server
including the complete password file, 
2) uses only software changes on the server (no hardware changes, 
additional servers, or client changes), and 3) requires exponentially more 
effort for the attacker.   
As such, PolyPassHash increases the difficulty
of cracking passwords while remaining easy to deploy.

