\section{Threat Model}
\label{sec-threatmodel}

Suppose that a server wants to validate passwords for user accounts.   
%registering an account, the user and server will agree on a salted 
%will register an account  provide the server with a salted hash of 
%the password.   
%Note that it is irrelevant which party generates the sal(In practice this is more commonly done on the server side, but 
%the salt generation and hash can be performed by either party.)
We assume that:

\begin{itemize}

\item An attacker can read all data that is persisted on disk including
the password file.

\item The server can fail and need to be restarted by administrators.   All
state that is kept in memory is lost at this point.   The server must
restart using only the (attacker visible) data from disk.   (While the basic
technique needs a threshold of correct passwords to perform verification, 
Section~\ref{sec-partial} discusses an extension to immediately verify
passwords upon restart.)

\item The attacker can have a priori knowledge of the correct
passwords for some number of user accounts.   We assume
that the administrator has configured the threshold to be larger than this
value.   (Note that a technique to have protected user accounts that
do not count toward the threshold is described in 
Section~\ref{sec-thresholdless}.)

\item 
The attacker {\bf cannot} read arbitrary memory for all processes on the 
server.  
If an attacker can read arbitrary memory, then the attacker can observe
the plain-text passwords as they are entered and so any form of secure password
storage can be bypassed.

The attacker cannot read arbitrary memory in many types of reported attacks
including compromises of password file data stored on 
backup media, web server misconfiguration, or information obtained through SQL 
injection attacks --- the primary vector for password file 
compromises~\cite{passwordresearchblog, miranteTR13}.

\item
Even if these guarantees are not met, the goal is to still retain
strong security.   Regardless of the attacker's knowledge, the protections
provided in this work must never be worse than the best practices used today 
(salted hashes of passwords).

\end{itemize}

Limitations and strengths of alternative solutions that take this threat 
model into account are discussed in Section~\ref{sec-expecteduse}.
